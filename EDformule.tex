%&LaTeX
\documentclass[11pt,twoside]{article}
\usepackage[left=0.25in,right=0.25in,top=0.25in,bottom=0.25in]{geometry} 
\geometry{paperwidth=10.625in,paperheight=13.75in}  
\setlength{\paperheight}{13.75in}
\setlength{\paperwidth}{10.625in}
              
\usepackage{graphicx}
\usepackage{amssymb}
\usepackage{epstopdf}
\usepackage{amsmath}
\usepackage{physics} % ovo je potrebno instalirati ako zelite da kompajlirate ovo
\usepackage{tensor}

\usepackage{textcomp}
\usepackage{fancyhdr}
\usepackage{hyperref}
\hypersetup{
    pdftitle={Formule iz elektrodinamike},    
    pdfauthor={Bukva Aleksandar},    
    pdfsubject={Formule iz elektrodinamike},   
    pdfcreator={pdftex},   
    pdfproducer={Texmaker}, 
    colorlinks=true,       
    linkcolor=red,         
    citecolor=red,       
    filecolor=red,     
    urlcolor=red           
}

\usepackage{multicol}

\date{}

\begin{document}

\begin{center}
\LARGE
Formule za elektrodinamiku \footnote{ \textcopyleft \ 2016. From  \url{https://github.com/abukva/edformule}, last revised \today.}
\end{center}
\normalsize
\begin{multicols}{3}

\begin{footnotesize}

\begin{equation}
\rho(\vb{r}) = \sum_\alpha q_a \delta^{(3)}(\vb{r}-\vb{r_\alpha})
\end{equation}

\begin{equation}
\div\vec{j} + \frac{\partial \rho}{\partial t} = 0
\end{equation}

\begin{equation}
\begin{split}
\vb{F} &= q(\vb{E} + \vb{v}\cross\vb{B}) \\
\vb{F} &= \int \rho \dd[3]{r}(\vb{E} + \vb{v}\cross\vb{B}) \\
&= \int (\rho\vb{E} + \vb{j}\cross\vb{B})\dd[3]{r}
\end{split}
\end{equation}

\begin{equation}
\vb{p} = \int_V \rho(\vb{r}) \vb{r} \dd V
\end{equation}

\begin{equation}
\phi = \frac{1}{4 \pi \epsilon_0} \int_V \frac{\rho(\vb{r'})}{|\vb{r}-\vb{r'}|}\dd V'
\end{equation}

\begin{equation}
\phi = \frac{1}{4 \pi \epsilon_0} \left [\frac{Q}{r} + \frac{\vb{r}\cdot\vb{p}}{r^3} + \frac{x_i D_{ij}x_j}{2r^5} \right ]
\end{equation}

\begin{equation}
\vb{E}(\vb{r}) = \frac{1}{4 \pi \epsilon_0} \int \frac{\rho(\vb{r}')\dd[3]{r}'}{\abs{\vb{r}-\vb{r}'}}(\vb{r}-\vb{r}')
\end{equation}

\begin{equation}
\begin{split}
&D_{ij} = \int \rho(\vb{r})(3x_ix_j-\delta_{ij}r^2)\dd V \\
&D_{ij} = \sum_{\alpha} q_{\alpha} ( 3 x_i^{(\alpha)}x_j^{(\alpha)} - \delta_{ij}(r^{(\alpha)})^2); \quad \Tr\hat{D} =0
\end{split}
\end{equation}

\begin{equation}
\vb{E} = - \gradient \phi
\end{equation}

\begin{equation}
\vb{j} = \rho \vb{v}
\end{equation}

\begin{equation}
\vb{A}(\vb{r}) = \frac{\mu_0}{4\pi} \int \frac{\vb{j}(\vb{r'})}{|\vb{r}-\vb{r'}|}\dd V'
\end{equation}

\begin{equation}
\vb{A}(\vb{r}) = \frac{\mu_0}{4\pi} \frac{\vb{m}\times\vb{r}}{r^3}
\end{equation}

\begin{equation}
\vb{m}(\vb{r}) = \frac{1}{2} \int \vb{r} \times \vb{j}(\vb{r})\dd V
\end{equation}

\begin{equation}
\vb{B}(\vb{r}) = \curl \vb{A} = \frac{\mu_0}{4\pi} \int \frac{\vb{j}(\vb{r'})\times (\vb{r}-\vb{r'})}{|\vb{r}-\vb{r'}|^3} \dd V'
\end{equation}

\begin{equation}
\vb{E} = -\pdv{\vb{A}}{t} - \gradient \phi
\end{equation}

\begin{equation}
\begin{split}
\div{\vb{A}} + \frac{1}{c^2} \pdv{\phi}{t} &= 0 \\
\laplacian{\phi} - \frac{1}{c^2} \pdv[2]{\phi}{t} &= -\frac{\rho}{\epsilon_0} \\
\laplacian{\vb{A}} - \frac{1}{c^2} \pdv[2]{\vb{A}}{t} &= -\mu_0 \vb{j}
\end{split}
\end{equation}

\begin{equation}
\expval{\rho} = \frac{1}{T} \int_0^T \rho dt 
\end{equation}

\begin{equation}
\laplacian \phi - \frac{1}{c^2} \frac{\partial^2 \phi}{\partial t^2} = -\frac{1}{\epsilon_0}\rho
\end{equation}

\begin{equation}
\laplacian \vb{A} - \frac{1}{c^2} \frac{\partial^2 \vb{A}}{\partial t^2} = -\mu_0 \vb{j}
\end{equation}

\begin{equation}
\begin{split}
D_{2n} - D_{1n} &= \sigma \\
B_{2n} - B_{1n} &= 0 \\
P_{2n} - P_{1n} &= -\sigma_{vez} \\
E_{2n} - E_{1n} &= \frac{1}{\epsilon_0} \sigma \\
E_{2t} - E_{1t} &= 0 \\
\vb{n}\times (\vb{H}_2 - \vb{H}_1) &= \vb{i} \\
M_{2t} - M_{1t} &= \vb{i}_{vez} \times \vb{n}
\end{split}
\end{equation}

\begin{equation}
\begin{split}
\div\vec{E} &= \frac{\rho}{\epsilon_0} \\
\div\vec{B} &= 0 \\
\curl \vb{E} &= -\frac{\partial\vb{B}}{\partial t} \\
\curl \vb{B} &= \mu_0 \left( \vb{j} + \epsilon_0 \frac{\partial\vb{E}}{\partial t} \right)
\end{split}
\end{equation}

\begin{equation}
\begin{split}
\div\vec{D} &= \rho + \rho_{ext} \\
\div\vec{B} &= 0 \\
\curl \vb{E} &= - \frac{\partial B}{\partial t} \\
\curl \vb{H} &= \vb{j} + \vb{j}_{ext} + \frac{\partial D}{\partial t}
\end{split}
\end{equation}

\begin{equation}
q = U C = U \epsilon_0 \frac{S}{d}
\end{equation}

\begin{equation}
\begin{split}
&\frac{d}{dt} \left( \sum_n \varepsilon_n + \underbrace{\int_V \dd[3]{r} \left( \frac{1}{2} \vb{D}\vdot\vb{E} + \frac{1}{2}\vb{H}\vdot\vb{B} \right)}_{W_{em}} \right) \\ &= -\oint_{\partial S} \vb{S}_p d\vb{S} \\
&\dv{t} (\sum_n \varepsilon_n ) = \int \dd[3]{r}\vb{j}\vdot\vb{E}
\end{split}
\end{equation}

\begin{equation}
\begin{split}
\vb{S}_p &= \vb{E}\cross\vb{H} \\
\vb{g} &= \frac{\vb{S}_p}{c^2} \quad \text{- gustina impulsa}
\end{split}
\end{equation}

\begin{equation}
\vb{L} = \int \vb{r} \times \vb{g} \dd V
\end{equation}

\begin{equation}
\begin{split}
\frac{d}{dt}\left[\sum_{\alpha} \vb{p}_{\alpha} + \int \dd[3]{r} (\vb{D} \times \vb{B}) \right] \\ - \frac{1}{2•}\int \dd[3]{r}(\epsilon_0 \vb{E}^2 \gradient \epsilon_r + \mu_0\vb{H}^2 \gradient \mu_r) = \oint_S \hat{T}d\vb{S} \\
\hat{T} = \dyad{\vb{E}}{\vb{D}} + \dyad{\vb{H}}{\vb{B}} - \frac{1}{2•}(\vb{E}\cdot\vb{D}+\vb{H}\cdot\vb{B})\hat{I}
\end{split}
\end{equation}

\begin{equation}
V^{\mu} = \mqty(\frac{c}{\sqrt{1-\frac{v^2}{c^2}}}\\\frac{\vb{v}}{\sqrt{1-\frac{v^2}{c^2}}})
\end{equation}

\begin{equation}
P^{\mu} = mV^{\mu} = \mqty(\frac{E}{c}\\\vb{p})
\end{equation}

\begin{equation}
\mathcal{A}^\mu = \dv{V^{\mu}}{\tau} = \mqty(c\gamma\dv{\gamma}{t}\\\gamma\dv{\gamma}{t}\vb{v}+\gamma^2\vb{a})
\end{equation}

\begin{equation}
\mathcal{F}^\mu = \mqty(\frac{\vb{v}\vdot\vb{F}}{c\sqrt{1-\frac{v^2}{c^2}}}\\\frac{\vb{F}}{\sqrt{1-\frac{v^2}{c^2}}})
\end{equation}

\begin{equation}
\begin{split}
\vb{E'}_{\parallel} &= \vb{R}_{\parallel} \\
\vb{E'}_{\perp} &= \frac{\vb{E}_{\perp}+\vb{v}\cross\vb{B}_{\perp}}{\sqrt{1-\frac{v^2}{c^2}}} \\
\vb{B'}_{\parallel} &= \vb{B}_{\parallel} \\
\vb{B'}_{\perp} &= \frac{\vb{B}_{\perp} - \frac{1}{c^2}\vb{v}\cross\vb{E}_{\perp}}{\sqrt{1-\frac{v^2}{c^2}}}
\end{split}
\end{equation}

\begin{equation}
\begin{split}
\vb{E'} &= \gamma(\vb{E} + \vb{v}\cross\vb{B}) - \frac{\gamma^2}{c^2(1+\gamma)}(\vb{v}\vdot\vb{E})\vb{v} \\
\vb{B'} &= \gamma(\vb{E} + \frac{1}{c^2}\vb{v}\cross\vb{B}) - \frac{\gamma^2}{c^2(1+\gamma)}(\vb{v}\vdot\vb{B})\vb{v}
\end{split}
\end{equation}

\begin{equation}
\mqty(x'^0 \\ x'^1 \\ x'^2 \\ x'^3) = \underbrace{\mqty(\gamma&-\beta\gamma&0&0 \\ -\beta\gamma&\gamma&0&0 \\ 0&0&1&0 \\ 0&0&0&1)}_{\Lambda\indices{^\mu_\nu}}\mqty(x^0 \\ x^1 \\ x^2 \\ x^3)
\end{equation}

\begin{equation}
g_{\mu\nu} = \mqty(\dmat[0]{1,-1,-1,-1})
\end{equation}

\begin{equation}
\pdv{x\indices{^\mu}} = \left(\frac{1}{c}\pdv{t}, \gradient \right)
\end{equation}

\begin{equation}
j\indices{^\mu} = (c\rho, \vb{j} = \rho \vb{v})
\end{equation}

\begin{equation}
A\indices{^\mu} = (\frac{\phi}{c}, \vb{A})
\end{equation}

\begin{equation}
\Box A\indices{^\mu} = \mu_0 j\indices{^\mu}
\end{equation}

\begin{equation}
F\indices{^{\mu\nu}} = \partial\indices{^\mu}A\indices{^\nu} - \partial\indices{^\nu}A\indices{^\mu} 
\end{equation}

\begin{equation}
F\indices{^{\mu\nu}} = \mqty(0&-\frac{E_x}{c}&-\frac{E_y}{c}&-\frac{E_z}{c} \\ \frac{E_x}{c}&0&-B_z&B_y \\ \frac{E_y}{c}&B_z&0&-B_x\\\frac{E_z}{c}&-B_y&B_x&0)
\end{equation}

\begin{equation}
F\indices{^{'\mu\nu}} = (\Lambda F \Lambda^T)\indices{^{\mu\nu}}
\end{equation}

\begin{equation}
\begin{split}
\dv{\vb{p}}{t}&=q(\vb{E}+\vb{v}\cross\vb{B})\\
\dv{t}\underbrace{\left(\frac{mc^2}{\sqrt{1-\frac{v^2}{c^2}}} \right)}_{E} &= \vb{v}\vdot\vb{F} = q\vb{v}\vdot\vb{E}
\end{split}
\end{equation}

\begin{equation}
E^2 = c^2p^2+m^2c^4
\end{equation}

\begin{equation}
\dv{\tau}\left(\pdv{\tilde{L}}{u\indices{^\alpha}} \right) - \pdv{\tilde{L}}{x\indices{^\alpha}} = 0
\end{equation}

\begin{equation}
\begin{split}
\pdv{u\indices{^\mu}}{u\indices{^\alpha}}&=\delta\indices{^\mu_\alpha} \\
\pdv{u\indices{^\mu}}{u\indices{_\alpha}}&=g\indices{^{\mu\alpha}}
\end{split}
\end{equation}

\begin{equation}
\begin{split}
S &= \int (-mc\dd s - qA\indices{^\mu}\dd x\indices{_\mu}) + S\indices{_f} \\
\dd s &= \sqrt{g\indices{_{\mu\nu}}\dv{x\indices{^\mu}}{\tau}\dv{x\indices{^\nu}}{\tau}} \dd \tau \\
\dd s &= c \dd \tau = c \sqrt{1-\frac{v^2}{c^2}}\dd t \\
A\indices{^\mu}\dd x\indices{_\mu} &= (\phi - \vb{A}\vdot\vb{v})\dd t
\end{split}
\end{equation}

%\columnbreak

\begin{equation}
\begin{split}
\sin(\alpha\pm\beta)&=\sin\alpha\cos\beta\pm\cos\alpha\sin\beta \\
\cos(\alpha\pm\beta)&=\cos\alpha\cos\beta\mp\sin\alpha\sin\beta \\
\tan(\alpha\pm\beta)&=\frac{\tan\alpha\pm\tan\beta}{1-\tan\alpha\tan\beta} \\
\cot(\alpha\pm\beta)&=\frac{\cot\alpha\cot\beta-1}{\cot\beta\pm\cot\alpha}
\end{split}
\end{equation}

\begin{equation}
\begin{split}
\sin(2\alpha)&=2\sin\alpha\cos\alpha \\
\cos(2\alpha)&=\cos^2\alpha -\sin^2\alpha \\
\tan(2\alpha)&=\frac{2\tan\alpha}{1-\tan^2\alpha}\\
\cot(2\alpha)&=\frac{\cot^2\alpha - 1}{2\cot\alpha}
\end{split}
\end{equation}

\begin{equation}
\begin{split}
\sin\alpha\pm\sin\beta&=2\sin\frac{\alpha\pm\beta}{2}\cos\frac{\alpha\mp\beta}{2} \\
\cos\alpha+\cos\beta&=2\cos\frac{\alpha+\beta}{2}\cos\frac{\alpha-\beta}{2} \\
\cos\alpha-\cos\beta&=-2\sin\frac{\alpha+\beta}{2}\sin\frac{\alpha-\beta}{2} \\
\tan\alpha \pm \tan\beta &= \frac{\sin(\alpha\pm\beta)}{\cos\alpha\cos\beta} \\
\cot\alpha \pm \cot\beta &= \frac{\sin(\alpha\pm\beta)}{\sin\alpha\sin\beta} \\
\sin\alpha\cos\beta&=\frac{1}{2}(\sin(\alpha+\beta)+\sin(\alpha-\beta)) \\
\cos\alpha\cos\beta&=\frac{1}{2}(\cos(\alpha+\beta)+\cos(\alpha-\beta)) \\
\sin\alpha\sin\beta &= -\frac{1}{2}(\cos(\alpha+\beta)+\cos(\alpha-\beta))
\end{split}
\end{equation}

\begin{equation}
\begin{split}
\sin\frac{\alpha}{2}&=\pm\sqrt{\frac{1-\cos\alpha}{2}} \\
\cos\frac{\alpha}{2}&=\pm\sqrt{\frac{1+\cos\alpha}{2}} \\
\tan\frac{\alpha}{2}&=\pm\sqrt{\frac{1-\cos\alpha}{1+\cos\alpha}}
\end{split}
\end{equation}

\end{footnotesize}
\end{multicols}
\end{document}  