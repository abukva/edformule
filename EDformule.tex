%&LaTeX
\documentclass[11pt,twoside]{article}
\usepackage[left=0.25in,right=0.25in,top=0.25in,bottom=0.25in]{geometry} 
\geometry{paperwidth=10.625in,paperheight=13.75in}  
\setlength{\paperheight}{13.75in}
\setlength{\paperwidth}{10.625in}
              
\usepackage{graphicx}
\usepackage{amssymb}
\usepackage{epstopdf}
\usepackage{amsmath}
\usepackage{physics}

\usepackage{textcomp}
\usepackage{fancyhdr}
\usepackage{hyperref}
\hypersetup{
    pdftitle={Formule iz elektrodinamike},    % title
    pdfauthor={Bukva Aleksandar},     % author
    pdfsubject={Formule iz elektrodinamike},   % subject of the document
    pdfcreator={pdftex},   % creator of the document
    pdfproducer={Texmaker}, % producer of the document
    colorlinks=true,       % false: boxed links; true: colored links
    linkcolor=red,          % color of internal links
    citecolor=red,        % color of links to bibliography
    filecolor=red,      % color of file links
    urlcolor=red           % color of external links
}

\usepackage{multicol}

\date{}

\begin{document}

\begin{center}
\LARGE
Formule za elektrodinamiku \footnote{\textcopyleft \ 2016. From  \url{https://github.com/abukva/edformule}, last revised \today. This material 
is provided as is without warranty or representation about the accuracy, correctness or suitability of the material for any purpose, and is licensed under the Creative Commons Attribution-Noncommercial-ShareAlike 3.0 United States License. To view a copy of this license, visit \url{http://creativecommons.org/licenses/by-nc-sa/3.0/} or send a letter to Creative Commons, 171 Second Street, Suite 300, San Francisco, California, 94105, USA. }
\end{center}
\normalsize
\begin{multicols}{3}

\begin{footnotesize}

\begin{equation}
\rho(\vb{r}) = \sum_\alpha q_a \delta^{(3)}(\vb{r}-\vb{r_\alpha})
\end{equation}

\begin{equation}
\div\vec{j} + \frac{\partial \rho}{\partial t} = 0
\end{equation}

\begin{equation}
\begin{split}
\vb{F} &= q(\vb{E} + \vb{v}\cross\vb{B}) \\
\vb{F} &= \int \rho \dd[3]{r}(\vb{E} + \vb{v}\cross\vb{B}) \\
&= \int (\rho\vb{E} + \vb{j}\cross\vb{B})\dd[3]{r}
\end{split}
\end{equation}

\begin{equation}
\vb{p} = \int_V \rho(\vb{r}) \vb{r} \dd V
\end{equation}

\begin{equation}
\phi = \frac{1}{4 \pi \epsilon_0} \int_V \frac{\rho(\vb{r\prime})}{|\vb{r}-\vb{r\prime}|}\dd V\prime
\end{equation}

\begin{equation}
\phi = \frac{1}{4 \pi \epsilon_0} \left [\frac{Q}{r} + \frac{\vb{r}\cdot\vb{p}}{r^3} + \frac{x_i D_{ij}x_j}{2r^5} \right ]
\end{equation}

\begin{equation}
\vb{E}(\vb{r}) = \frac{1}{4 \pi \epsilon_0} \int \frac{\rho(\vb{r}\prime)\dd[3]{r}\prime}{\abs{\vb{r}-\vb{r}\prime}}(\vb{r}-\vb{r}\prime)
\end{equation}

\begin{equation}
\begin{split}
&D_{ij} = \int \rho(\vb{r})(3x_ix_j-\delta_{ij}r^2)\dd V \\
&D_{ij} = \sum_{\alpha} q_{\alpha} ( 3 x_i^{(\alpha)}x_j^{(\alpha)} - \delta_{ij}(r^{(\alpha)})^2); \quad \Tr\hat{D} =0
\end{split}
\end{equation}

\begin{equation}
\vb{E} = - \gradient \phi
\end{equation}

\begin{equation}
\vb{j} = \rho \vb{v}
\end{equation}

\begin{equation}
\vb{A}(\vb{r}) = \frac{\mu_0}{4\pi} \int \frac{\vb{j}(\vb{r\prime})}{|\vb{r}-\vb{r\prime}|}\dd V\prime
\end{equation}

\begin{equation}
\vb{A}(\vb{r}) = \frac{\mu_0}{4\pi} \frac{\vb{m}\times\vb{r}}{r^3}
\end{equation}

\begin{equation}
\vb{m}(\vb{r}) = \frac{1}{2} \int \vb{r} \times \vb{j}(\vb{r})\dd V
\end{equation}

\begin{equation}
\vb{B}(\vb{r}) = \curl \vb{A} = \frac{\mu_0}{4\pi} \int \frac{\vb{j}(\vb{r\prime})\times (\vb{r}-\vb{r\prime})}{|\vb{r}-\vb{r\prime}|^3} \dd V\prime
\end{equation}

\begin{equation}
\vb{E} = \pdv{\vb{A}}{t} - \gradient \phi
\end{equation}

\begin{equation}
\begin{split}
\div{\vb{A}} + \frac{1}{c^2} \pdv{\phi}{t} &= 0 \\
\laplacian{\phi} - \frac{1}{c^2} \pdv[2]{\phi}{t} &= -\frac{\rho}{\epsilon_0} \\
\laplacian{\vb{A}} - \frac{1}{c^2} \pdv[2]{\vb{A}}{t} &= -\mu_0 \vb{j}
\end{split}
\end{equation}

\begin{equation}
\expval{\rho} = \frac{1}{T} \int_0^T \rho dt 
\end{equation}

\begin{equation}
\laplacian \phi - \frac{1}{c^2} \frac{\partial^2 \phi}{\partial t^2} = -\frac{1}{\epsilon_0}\rho
\end{equation}

\begin{equation}
\laplacian \vb{A} - \frac{1}{c^2} \frac{\partial^2 \vb{A}}{\partial t^2} = -\mu_0 \vb{j}
\end{equation}

\begin{equation}
\begin{split}
D_{2n} - D_{1n} &= \sigma \\
B_{2n} - B_{1n} &= 0 \\
P_{2n} - P_{1n} &= -\sigma_{vez} \\
E_{2n} - E_{1n} &= \frac{1}{\epsilon_0} \sigma \\
E_{2t} - E_{1t} &= 0 \\
\vb{n}\times (\vb{H}_2 - \vb{H}_1) &= \vb{i} \\
M_{2t} - M_{1t} &= \vb{i}_{vez} \times \vb{n}
\end{split}
\end{equation}

\begin{equation}
\begin{split}
\div\vec{E} &= \frac{\rho}{\epsilon_0} \\
\div\vec{B} &= 0 \\
\curl \vb{E} &= -\frac{\partial\vb{B}}{\partial t} \\
\curl \vb{B} &= \mu_0 \left( \vb{j} + \epsilon_0 \frac{\partial\vb{E}}{\partial t} \right)
\end{split}
\end{equation}

\begin{equation}
\begin{split}
\div\vec{D} &= \rho + \rho_{ext} \\
\div\vec{B} &= 0 \\
\curl \vb{E} &= - \frac{\partial B}{\partial t} \\
\curl \vb{H} &= \vb{j} + \vb{j}_{ext} + \frac{\partial D}{\partial t}
\end{split}
\end{equation}

\begin{equation}
q = U C = U \epsilon_0 \frac{S}{d}
\end{equation}

\begin{equation}
\begin{split}
&\frac{d}{dt} \left( \sum_n \varepsilon_n + \underbrace{\int_V \dd[3]{r} \left( \frac{1}{2} \vb{D}\vdot\vb{E} + \frac{1}{2}\vb{H}\vdot\vb{B} \right)}_{W_{em}} \right) \\ &= -\oint_{\partial S} \vb{S}_p d\vb{S} \\
&\dv{t} (\sum_n \varepsilon_n ) = \int \dd[3]{r}\vb{j}\vdot\vb{E}
\end{split}
\end{equation}

\begin{equation}
\begin{split}
\vb{S}_p &= \vb{E}\cross\vb{H} \\
\vb{g} &= \frac{\vb{S}_p}{c^2} \quad \text{- gustina impulsa}
\end{split}
\end{equation}

\begin{equation}
\vb{L} = \int \vb{r} \times \vb{g} \dd V
\end{equation}

\begin{equation}
\begin{split}
\frac{d}{dt}\left[\sum_{\alpha} \vb{p}_{\alpha} + \int \dd[3]{r} (\vb{D} \times \vb{B}) \right] \\ - \frac{1}{2•}\int \dd[3]{r}(\epsilon_0 \vb{E}^2 \gradient \epsilon_r + \mu_0\vb{H}^2 \gradient \mu_r) = \oint_S \hat{T}d\vb{S} \\
\hat{T} = \vb{\ket{E}}\vb{\bra{D}} + \vb{\ket{B}}\vb{\bra{H}} - \frac{1}{2•}(\vb{E}\cdot\vb{D}+\vb{H}\cdot\vb{B})I
\end{split}
\end{equation}

\end{footnotesize}
\end{multicols}
\end{document}  